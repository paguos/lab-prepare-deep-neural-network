% !TEX root = ../Projektdokumentation.tex
\clearpage
\section{Implementierungsphase} 
\label{sec:Implementierungsphase}

\subsection{Zielplattform}
\label{sec:Zielplattform}

\begin{itemize}
	\item Programmiersprachen: Python f\"ur NN, ggfs. JavaScript / TypeScript f\"ur UI
	\item UI lokal deployen oder via Server
	\item Hardware Platform: Linux
\end{itemize}
%
%\subsection{Implementierung der Datenstrukturen}
%\label{sec:ImplementierungDatenstrukturen}
%
%\begin{itemize}
%	\item Beschreibung der angelegten Datenbank (\zB Generierung von \acs{SQL} aus Modellierungswerkzeug oder händisches Anlegen), \acs{XML}-Schemas \usw.
%\end{itemize}
%
\subsection{Implementierung des NN}
\label{sec:ImplementierungNN}

\begin{itemize}
	\item Wahl eines sinnvollen NN (Gr\"o{\ss}e des NN muss m.H.d. LEDs abbildbar sein, einzelne Neuronen sollen stark aktiviert werden, sodass dies visuell f\"ur den Nutzer bemerkbar ist) 
	\item Implementierung des NN
	\item Sinnvolle Darstellung der Gewichte in der UI
	\item ggfs. zusammenfassen von Layern des NN
	\item Testen des NN mit verschiedenen Inputs (Audio / Bilder / Digits)
%	\item TODO: noch irgendwelche anderen Sachen @Philipp?
%	\item TODO: Screenshots der Anwendung
\end{itemize}


\subsection{Implementierung der Benutzeroberfläche}
\label{sec:ImplementierungBenutzeroberflaeche}

\begin{itemize}
	\item Programmierung der UI
	\item Implementierung der Schnittstellen zum Pi 
	\item Testen der UI Funktionalit\"aten (Anzeigen der richtigen Gewichte? Ggfs. User-Input? Formatierung in verschiedenen Browsern / Endger\"aten)
%	\item TODO: Screenshots der Anwendung
\end{itemize}



\subsection{Aufgabenaufteilung}
\label{sec:aufgabenaufteilung}

\begin{itemize}
	\item Implementierung des NN: Philipp
	\item UI \& APIs: Annabella
	\item Hardware, Raspberry Pi -- LED Programmierung: Pablo
\end{itemize}


\subsection{Meilensteine}
\label{sec:meilensteine}

\begin{itemize}
	\item MS1: Beschaffung aller Hardware Materialien \& grobe Implementierung des NN (bis 30.06.20)
	\item MS2: Implementierung aller APIs \& Testen des Zusammenspiels aller Komponenten (bis 15.08.20)
	\item MS3: Optimierung des NN, ggfs. Erweiterung auf weitere Anwendungsf\"alle, Dokumentation, Projektbericht (bis 20.09.20)
\end{itemize}

%\paragraph{Beispiel}
%Screenshots der Anwendung in der Entwicklungsphase mit Dummy-Daten befinden sich im \Anhang{Screenshots}.
%
%
%\subsection{Implementierung der Geschäftslogik}
%\label{sec:ImplementierungGeschaeftslogik}
%
%\begin{itemize}
%	\item Beschreibung des Vorgehens bei der Umsetzung/Programmierung der entworfenen Anwendung.
%	\item \Ggfs interessante Funktionen/Algorithmen im Detail vorstellen, verwendete Entwurfsmuster zeigen.
%	\item Quelltextbeispiele zeigen.
%	\item Hinweis: Wie in Kapitel~\ref{sec:Einleitung}: \nameref{sec:Einleitung} zitiert, wird nicht ein lauffähiges Programm bewertet, sondern die Projektdurchführung. Dennoch würde ich immer Quelltextausschnitte zeigen, da sonst Zweifel an der tatsächlichen Leistung des Prüflings aufkommen können.
%\end{itemize}
%
%\paragraph{Beispiel}
%Die Klasse \texttt{Com\-par\-ed\-Na\-tu\-ral\-Mo\-dule\-In\-for\-ma\-tion} findet sich im \Anhang{app:CNMI}.  
