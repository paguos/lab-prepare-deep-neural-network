% !TEX root = ../Projektdokumentation.tex
\section{Einleitung}
\label{sec:Einleitung}


%\subsection{Projektumfeld} 
%\label{sec:Projektumfeld}
%\begin{itemize}
%	\item Kurze Vorstellung des Ausbildungsbetriebs (Geschäftsfeld, Mitarbeiterzahl \usw)
%	\item Wer ist Auftraggeber/Kunde des Projekts?
%\end{itemize}
%
%
\subsection{Projektziel} 
\label{sec:Projektziel}
\begin{itemize}
	\item Visualisierung eines neuronalen Netzes (NN)
	\item Zielgruppe: Menschen ohne Machine Learning (ML) Vorkenntnisse 
	\item Input: Zahlen (Digit-Recognition)
	\item Visualisierung der Neuronengewichte m.H.e. UI (Python / JavaScript / TypeScript basiert)
\end{itemize}


\subsection{Projektbegründung} 
\label{sec:Projektbegruendung}
\begin{itemize}
	\item NN sind oftmals f\"ur Menschen ohne IT-Hintergrund schwer verst\"andlich
	\item Die Interaktive Darstellung erlaubt Menschen ohne ML Kenntnisse, dieses Thema zumindest in den Grundz\"ugen besser zu verstehen
\end{itemize}	

%
%\subsection{Projektschnittstellen} 
%\label{sec:Projektschnittstellen}
%\begin{itemize}
%	\item Mit welchen anderen Systemen interagiert die Anwendung (technische Schnittstellen)?
%	\item Wer genehmigt das Projekt \bzw stellt Mittel zur Verfügung? 
%	\item Wer sind die Benutzer der Anwendung?
%	\item Wem muss das Ergebnis präsentiert werden?
%\end{itemize}


\subsection{Projektabgrenzung} 
\label{sec:Projektabgrenzung}
\begin{itemize}
	\item Wir beschr\"anken uns auf kleinere NN, da jedes Neuron durch eine LED repr\"asentiert wird
	\item Anwendungsfall: Digit Recognition, Sound / Bilder Kategorisierung
	\item ggfs. nur \"Anderung der Input Gewichte, ansonsten reine Visualisierung 
\end{itemize}
