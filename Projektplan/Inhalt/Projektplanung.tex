% !TEX root = ../Projektdokumentation.tex
\clearpage
\section{Projektplanung} 
\label{sec:Projektplanung}


\subsection{Projektphasen}
\label{sec:Projektphasen}
Projektdauer: 01.05. -- 30.09.2020
%
%\begin{itemize}
%	\item 
%	\item Verfeinerung der Zeitplanung, die bereits im Projektantrag vorgestellt wurde.
%\end{itemize}

\paragraph{Beispiel}
Tabelle~\ref{tab:Zeitplanung} zeigt ein Beispiel für eine grobe Zeitplanung.
\tabelle{Zeitplanung}{tab:Zeitplanung}{ZeitplanungKurz}\\
Eine detailliertere Zeitplanung findet sich im \Anhang{app:Zeitplanung}.

%
%\subsection{Abweichungen vom Projektantrag}
%\label{sec:AbweichungenProjektantrag}
%
%\begin{itemize}
%	\item Sollte es Abweichungen zum Projektantrag geben (\zB Zeitplanung, Inhalt des Projekts, neue Anforderungen), müssen diese explizit aufgeführt und begründet werden.
%\end{itemize}
%

\subsection{Ressourcenplanung}
\label{sec:Ressourcenplanung}

\begin{itemize}
	\item Hardware: Raspberry Pi 3, LED Matrix / Streifen, Monitor (bereits vorhanden)
	\item LED Matrix oder LED Streifen (ggfs. vom lab:prepare Team oder bestellen)
	\item Ber\"ucksichtigung von Wartezeiten (Bestellungen) $\rightarrow$ Programmierung der Visualisierungs-UI trotzdem m\"oglich (reine Programmierung)
%	\item \Ggfs sind auch personelle Ressourcen einzuplanen (\zB unterstützende Mitarbeiter).
%	\item Hinweis: Häufig werden hier Ressourcen vergessen, die als selbstverständlich angesehen werden (\zB PC, Büro). 
\end{itemize}


\subsection{Entwicklungsprozess}
\label{sec:Entwicklungsprozess}
\begin{itemize}
	\item Agiler Entwicklungsprozess (je nach Arbeitsload durch die Uni)
	\item Hauptentwicklungsphase ab Juli, da wegen Corona w\"ahrend dem Semester Klausuren aus dem WiSe 20/21 nachgeschrieben werden m\"ussen
\end{itemize}
