%-----------------------------------------------------------------------------------------
% Autor dieser Vorlage:
% Stefan Macke (http://fachinformatiker-anwendungsentwicklung.net)
% Permalink zur Vorlage: http://fiae.link/LaTeXVorlageFIAE
%
% Sämtliche verwendeten Abbildungen, Tabellen und Listings stammen von Dirk Grashorn.
%
% Lizenz: Creative Commons 4.0 Namensnennung - Weitergabe unter gleichen Bedingungen
% -----------------------------------------------------------------------------------------

\documentclass[
	ngerman,
	toc=listof, % Abbildungsverzeichnis sowie Tabellenverzeichnis in das Inhaltsverzeichnis aufnehmen
	toc=bibliography, % Literaturverzeichnis in das Inhaltsverzeichnis aufnehmen
	footnotes=multiple, % Trennen von direkt aufeinander folgenden Fußnoten
	parskip=half, % vertikalen Abstand zwischen Absätzen verwenden anstatt horizontale Einrückung von Folgeabsätzen
	numbers=noendperiod % Den letzten Punkt nach einer Nummerierung entfernen (nach DIN 5008)
]{scrartcl}
\pdfminorversion=5 % erlaubt das Einfügen von pdf-Dateien bis Version 1.7, ohne eine Fehlermeldung zu werfen (keine Garantie für fehlerfreies Einbetten!)
\usepackage[utf8]{inputenc} % muss als erstes eingebunden werden, da Meta/Packages ggfs. Sonderzeichen enthalten

\input{Meta} % Metadaten zu diesem Dokument (Autor usw.)
\input{Allgemein/Packages} % verwendete Packages
\input{Allgemein/Seitenstil} % Definitionen zum Aussehen der Seiten
\input{Allgemein/Befehle} % eigene allgemeine Befehle, die z.B. die Arbeit mit LaTeX erleichtern
\input{Befehle} % eigene projektspezifische Befehle, z.B. Abkürzungen usw.

\begin{document}

% kann nach dem Lesen entfernt werden ---------------------------------------
%\pagestyle{plain}
%\input{Vorlage}
%\cleardoublepage
%\pagestyle{scrheadings}
% ---------------------------------------------------------------------------

\phantomsection
\thispagestyle{empty}
%\pdfbookmark[1]{Eidesstattliche Erklärung}{ihkdeckblatt}
%\includegraphicsKeepAspectRatio{DeckblattIHK}{1}
%\cleardoublepage

\phantomsection
\thispagestyle{plain}
%\pdfbookmark[1]{Deckblatt}{deckblatt}
\input{Deckblatt}
\cleardoublepage

% Preface --------------------------------------------------------------------
\phantomsection
\pagenumbering{Roman}
\pdfbookmark[1]{Inhaltsverzeichnis}{inhalt}
%\tableofcontents
%\cleardoublepage
%
%\phantomsection
%\listoffigures
%\cleardoublepage
%
%\phantomsection
%\listoftables
%\cleardoublepage
%
%\phantomsection
%\lstlistoflistings
%\cleardoublepage
%
%\newcommand{\abkvz}{Abkürzungsverzeichnis}
%\renewcommand{\nomname}{\abkvz}
%\section*{\abkvz}
%\markboth{\abkvz}{\abkvz}
%\addcontentsline{toc}{section}{\abkvz}
%\input{Abkuerzungen}
\clearpage

% Inhalt ---------------------------------------------------------------------
\pagenumbering{arabic}
\input{Inhalt.tex}

% Literatur ------------------------------------------------------------------
%\clearpage
%\renewcommand{\refname}{Literaturverzeichnis}
%\bibliography{Bibliographie}
%\bibliographystyle{Allgemein/natdin} % DIN-Stil des Literaturverzeichnisses
%\input{Erklaerung}

% Anhang ---------------------------------------------------------------------
\clearpage
\appendix
\pagenumbering{roman}
\input{Anhang}

\end{document}
